\chapter{Listings}
This appendix provides the actual C code for some implemented functions discussed in Chapter~\ref{chp:implementation}.


\section{\_GetRandomLevel function}
\label{sec:GetRandomLevel}
The \texttt{\_GetRandomLevel} function has been introduced in Section~\ref{sec:FastRNG}.
It implements the Random Number Generator required for determining the level of a new element added to a Skip List.
\bigskip

\begin{lstlisting}
// Define SKIPLIST_LEVELS to the maximum number of levels the Skip List shall have.
#define SKIPLIST_LEVELS    16

C_ASSERT(SKIPLIST_LEVELS >= 1);
C_ASSERT(SKIPLIST_LEVELS <= 31);

static __inline CHAR
_GetRandomLevel()
{
    // Using a simple fixed seed and the Park-Miller Lehmer Minimal Standard Random Number Generator gives an acceptable distribution for our "random" levels.
    static DWORD dwRandom = 1;

    DWORD dwLevel = 0;
    DWORD dwShifted;

    // Generate 31 uniformly distributed pseudo-random bits using the Park-Miller Lehmer Minimal Standard Random Number Generator.
    dwRandom = (DWORD)(((ULONGLONG)dwRandom * 48271UL) % 2147483647UL);

    // Shift out (31 - SKIPLIST_LEVELS) bits to the right to have no more than SKIPLIST_LEVELS bits set.
    dwShifted = dwRandom >> (31 - SKIPLIST_LEVELS);

    // BitScanForward doesn't operate on a zero input value.
    if (dwShifted)
    {
        // BitScanForward sets dwLevel to the zero-based position of the first set bit (from LSB to MSB).
        // This makes dwLevel a geometrically distributed value between 0 and SKIPLIST_LEVELS - 1 for p = 0.5.
        BitScanForward(&dwLevel, dwShifted);
    }

    // dwLevel can't have a value higher than 30 this way, so a CHAR is more than enough.
    return (CHAR)dwLevel;
}
\end{lstlisting}


\section{\_RpcWritePrinter implementation}
\label{sec:RpcWritePrinter}
\texttt{\_RpcWritePrinter} is one of the \gls{RPC} server functions implemented in the Spooler Server.
It serves as an example how most of the \gls{RPC} calls are implemented.
Basically, Impersonation is performed, the corresponding Spooler Router function called, and finally the security context reverted back to the system context.
This process is further illustrated in Figure~\ref{fig:WritePrinterProcessing}.
\bigskip

\begin{lstlisting}
DWORD
_RpcWritePrinter(WINSPOOL_PRINTER_HANDLE hPrinter, BYTE* pBuf, DWORD cbBuf, DWORD* pcWritten)
{
    DWORD dwErrorCode;

    dwErrorCode = RpcImpersonateClient(NULL);
    if (dwErrorCode != ERROR_SUCCESS)
    {
        ERR("RpcImpersonateClient failed with error %lu!\n", dwErrorCode);
        return dwErrorCode;
    }

    WritePrinter(hPrinter, pBuf, cbBuf, pcWritten);
    dwErrorCode = GetLastError();

    RpcRevertToSelf();
    return dwErrorCode;
}
\end{lstlisting}
