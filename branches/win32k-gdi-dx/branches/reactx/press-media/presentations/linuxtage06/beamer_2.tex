\documentclass{beamer}

\usetheme{Warsaw}
\usepackage[english]{babel}
\usepackage[latin1]{inputenc}
\usepackage{listings}
\usepackage{fancyvrb}
\usepackage{graphicx}

\setbeamercovered{transparent}

\title{ReactOS@LinuxTage06}
\author{Johannes Anderwald, Alexander Wurzinger, Martin Rottensteiner}
%\institute{Technische Universit�t Graz}

\begin{document}
\frametitle{ReactOS@LinuxTage06}
\nameslide{Einf�hrung in ReactOS}
\frame{\titlepage}

%----------------------------------------------- new chapter ---------------------
\frame{
\section*{Was ist ReactOS}
\begin{itemize}
  \item {Geschichte von ReactOS}
  \item {Warum ReactOS}
  \item {Windows NT Architektur}
  \item {Unterst�tzte Architekturen}
  \item {Hardwareanforderungen}
\end{itemize}
}

\subsection{Geschichte von ReactOS}
\frame{
  \frametitle {Geschichte von ReactOS}
  \begin{itemize}
    \item {1996 Versuch mit "FreeWin95"}
    \item {1998 Neustart mit NT Kernel als Ziel}
    \item {2003 0.1.0 - ohne graphische Oberfl�che}
    \item {2004 0.2.0: mit graphischer Oberfl�che}
    \item {2005 Openoffice, Unreal Tournament, mIRC, }
\end{itemize}
}

\subsection{Warum ReactOS}
\frame{
   \frametitle{Warum ReactOS}
   \begin{itemize}
      \item {Open Source GPL License}
      \item {Kompatibel mit Microsoft Windows Programmen \& Treibern}
      \item {Microsoft Monopol zu brechen}
      \item {Ungefixte Sicherheitsl�cken in Windows}
      \item {Dokumentation f�r Applikationsentwickler / Treiberentwickler}
   \end{itemize}
}


\subsection{Windows NT Architektur}
\frame{
  \frametitle{Windows NT Architektur}
  \begin{itemize}
     \item {Modifizierter Microkernel}
     \item {Unterschiedliche Privilegienmodus}
        \begin{itemize}
         \item{Kernel, Treiber $\rightarrow$ Ring 0}
         \item{User Programme, Shared DLL $\rightarrow$ Ring 4}
        \end{itemize}
     \item {Kommunikation von User-Kernelmodus �ber NCI (Native Call Interface)}
        \begin{itemize}
           \item{movl 0x1000, eax}
           \item{movl KUSER\_SHARED\_SYSCALL, ecx}
           \item{call *ecx}
           \item{ret 0x4}
        \end{itemize}
     \item {Robust, skalierbar, erweiterbar}        
  \end{itemize}
}

\subsection{Unterst�tzte Architekturen}
\frame{
  \frametitle{Unterst�zte Architekturen}
\begin{itemize}
  \item{Intel x86}
    \begin{itemize}
       \item {i386 wird nicht unterst�tzt}
       \item {Keine Instruktionen f�r atomares Lesen/Schreiben verf�gbar}
       \item {Ab i486 mit Instruktion \textbf{cmpxchg} realisiert}
    \end{itemize}
  \item{Unterst�tzt SMP (Symmetric Multiprocessor Systems)}
  \item{Xen Port}
  \item{PPC Port in Planungstadium}
\end{itemize}
}



\subsection{Hardwareanforderungen}
\frame{
  \frametitle{Hardwareanforderungen}
  \begin{itemize}
    \item{32 MB RAM (64MB f�r Installation), 24MB f�r Textmodus}
    \item {IDE Festplatte mit mindestens 100MB freien Speicher}
    \item {VGA kompatible Grafikkarte}
    \item {PS/2 Maus + Tastatur empfohlen}
    \item {USB Maus + Tastatur noch nicht stabil}
\end{itemize}
}

%----------------------------------------------- new chapter ---------------------
%\section{ReactOS Installation}
%\frame{
%  \frametitle{ReactOS Installation}
%  \begin{itemize}
%    \item{Echter Hardware}
%      \begin{itemize}
%         \item{Bootet derzeit nur von FAT16/24/32 Partitionen}
%         \item{Partitionieren ist noch sehr instabil!}
%         \item{Backup machen bevor ReactOS Installation!}
%         \item{Oder: Testen mit Live-CD}
%      \end{itemize}
%  \end{itemize}
%}
%\subsection{Virtual Computing}
%\frame{
%  \frametitle {Virtual Computing}
%\begin{itemize}
%  \item{Emulatoren}
%   \begin{itemize}
%      \item {QEmu - OpenSource Emulator funktioniert recht gut}
%      \item {VmWare Player - gratis aber mit Einschr�nkungen}
%   \end{itemize}
%   \item{Mehr Informationen auf } % FIXME\href{http:\\wiki.reactos.org}{ReactOS Wiki}}
%\end{itemize}
%}

%----------------------------------------------- new chapter ---------------------
  \section{Entwicklungswerkzeuge \& Buildprocess}
\frame{
  \frametitle{Entwicklungstools \& Buildprocess}
  \begin{itemize}
    \item{ReactOS Code Repository}
    \item{ReactOS RBuild System}
    \item{ReactOS Build Environment}
    \item{ReactOS Build Target}
  \end{itemize}
}

\subsection{ReactOS Code Repository}
\frame{
   \frametitle{ReactOS Code Repository}
   \begin{itemize}
      \item {ReactOS Codebase wird von Subversion verwaltet}
      \item {Subversion Clients http://subversion.tigris.org/}
      \item {TortoiseSVN http://tortoisesvn.tigris.org/ - Integrierbar in den Windows Explorer}
      \item {Checkout mit svn co svn://svn.reactos.org/reactos/trunk/reactos}
      \item {Unterschiedliche Branches sind unter svn://svn.reactos.org/reactos/ verf�gbar}
   \end{itemize}
}

\subsection{ReactOS RBuild System}
\frame{
  \frametitle{ReactOS RBuild System}
  \begin{itemize}
    \item{RBuild (.rbuild) Build Dateien}
    \item {XML Syntax}
    \item {Erm�glicht einfaches Bearbeiten der Buildparameter \\
           zB.: \textless define name="\_\_USE\_W32API" / \textgreater \\
           definiert ein Pr�prozessormakro}
    \item {Automatische Makefile Generierung / Abh�ngigkeitstests}
    \item {Backends f�r DevCPP / MSVS 6.0/2002/2003/2005}
  \end{itemize}
}

\subsection{Empfohlene Compiler}
\frame{
  \frametitle{Empfohlene Compiler}
  \begin{itemize}
    \item{Prim�r: Mingw GCC}
      \begin {itemize}
        \item {Standard-GCC 3.4.3}
        \item {GCC 4.1.0 hat noch kleine Probleme}
      \end {itemize}
    \item{MSVC viele Header/Linking Probleme}
  \end{itemize}
}


\subsection{ReactOS Build Environment}
\frame{
  \frametitle{ReactOS Build Environment}
  \begin{itemize}
    \item{Windows: ReactOS Build Environment \\    
     http://blight.reactos.at/reactos-be/ReactOS\%20Build\%20Environment\%200.1-3.4.2.exe}
    \item{Linux: Mingw-Cross Compiler Skript \\
           http://www.mingw.org/MinGWiki/index.php/BuildMingwCross}
    \item {Debian: \textbf{sudo apt-get install subversion nasm mingw32 mingw32-binutils mingw32-runtime}}
    \item {Dokumentation f�r Gentoo/Linux64/FreeBSD unter http://www.reactos.org/wiki/index.php/Build\_Environment}
  \end{itemize}
}

\subsection{ReactOS Build Targets}
\frame{
  \frametitle{ReactOS Build Targets}
  \begin{itemize}
    \item{\textbf{mingw32-make} - kompiliert ReactOS}
    \item{\textbf{mingw32-make bootcd} - erstellt eine installierbare ISO Image von ReactOS}
    \item{\textbf{mingw32-make install} - aktualisiert eine ReactOS Installation \\
          ROS\_INSTALL="X:ReactOS"} % FIXME
    \item{\textbf{mingw32-make msvc6/msvc7/msvc71/msvc8} - erstellt VS Projektdateien}
    \item{Weitere Optionen in der Datei "Makefile" beschrieben}
  \end{itemize}
}

%----------------------------------------------- new chapter ---------------------

\section{ReactOS Entwicklungsprozess}
\frame{
  \frametitle{ReactOS Entwicklungsprozess}
  \begin{itemize}
    \item {Entwicklungsprozess}
    \item {N�tzliche Entwicklungstools}
    \item {Code Re-use}
    \item {ReactOS Testen}
    \item {ReactOS Debuggen}
  \end{itemize}
}

\subsection{ReactOS Entwicklungsprozess}
\frame{
  \frametitle{ReactOS Entwicklungsprozess}
  \begin{itemize}
      \item {Keine Entwicklungsteams}
      \item {Entwickler k�nnen sich Arbeitsbereich selbst aussuchen}
      \item {Hauptinformationsquellen}
       \begin{itemize}
         \item{MSDN Library http://msdn.microsoft.com}
         \item{ReactOS IRC irc://irc.freenode.net/reactos}
         \item{ReactOS Mailinglist reactos-dev@reactos (needs subscription)}
       \end{itemize}
      \item {Keine Dokumentation $\rightarrow$ Reverse Engineering} %
       \begin{itemize}
         \item {Clean-Room Reverse-Engineering}
         \item {Dirty-Room Reverse-Engineering}      
       \end{itemize}
  \end{itemize}
}

\subsection{N�tzliche Entwicklungstools}
\frame{
    \frametitle {N�tzliche Entwicklungstools}
    \begin{itemize}
       \item {Dependency-Walker http://www.dependencywalker.com/}
       \item {Resource-Hacker http://www.angusj.com/resourcehacker/}
       \item {XVI32 http://www.chmaas.handshake.de/}
       \item {IDA http://www.datarescue.com/idabase/overview.htm}
    \end{itemize}
}

\subsection{Code Re-use}
\frame{
  \frametitle{Code Re-use}
  \begin{itemize}
      \item {WINE (Wine Is Not an Emulator) 25\% of ReactOS User Mode Libraries}
      \item {FreeType (Graphic Rendering Library) f�r freetype.dll}
      \item {Linux USB Kernel Stack}
      \item {Oskittcp - FreeBSD TCP/IP Stack f�r tcpip.sys}
      \item {Viele weitere...}
  \end{itemize}
}

\subsection{Reactos Testen}
\frame{
   \frametitle {ReactOS Testen}
   \begin {itemize}
      \item {Programme/Usermode DLLs $\rightarrow$ ReactOS/Windows}
      \item {Ausnahmen: user32.dll/ntdll.dll/gdi32.dll}
      \item {Kernel/Treiber: $\rightarrow$ ReactOS}
      \item {Wine Regression Suite}
      \item {CIS (Continious Integration System)}
   \end{itemize}
}

\subsection{ReactOS Debuggen}
\frame{
  \frametitle{ReactOS Debuggen}
  \begin{itemize}
    \item {Serial Port Logging: /DEBUGPORT=COM1}
    \item {Kernel Debugger: Kdbg / Gdb }
    \item {Loggen mit DPRINT1(const char * fmt, ...);}
    \item {BSOD (Blue Screen of Death mit Tab-K)}
  \end{itemize}
}

%----------------------------------------------- new chapter ---------------------

\section{ReactOS Zukunft}
\frame {
   \frametitle{ReactOS Zukunft}
   \begin{itemize}
       \item{ReactOS Roadmap}
       \item{ReactOS Zukunft}
       \item{Screenshots}
   \end{itemize}
}

\subsection{ReactOS Roadmap}
\frame{
  \frametitle{ReactOS Roadmap}
  \begin{itemize}
    \item{0.3.0 - Stable Ethernet 802.3 Networking}
    \item{0.4.0 - PnP (Plug \& Play Support) + USB Unterst�tzung}
    \item{0.5.0 - NTFS / Ext2 Lesen \& Schreiben, DirectX}
    \item{1.0.0 - ReactOS einsetzbar als Arbeits/Entwicklungsbetriebssystem}
  \end{itemize}
}

\subsection{ReactOS Zukunft}
\frame {
   \frametitle{ReactOS Zukunft}
   \begin{itemize}
     \item {Release 0.3.0 demn�chst}
     \item {PnP bereits in der ReactOS Codebase}
   \end{itemize}
}
\subsection{Screenshots}
\frame{
   \frametitle{Screenshots}
   \begin{center}
   \includegraphics[width=0.6\linewidth]{abiword.jpg} \\
   Abiword 0.3.0-SVN
   \end{center}
}

\frame{
   \frametitle{Screenshots}
   \begin{center}
   \includegraphics[width=0.6\linewidth]{mirc.jpg} \\
   mIRC 0.3.0-SVN
   \end{center}
}

\frame{
   \frametitle{Screenshots}
   \begin{center}
   \includegraphics[width=0.6\linewidth]{putty.jpg} \\
   Putty 0.3.0-SVN
   \end{center}
}

\frame {
   \frametitle{Screenshots}
   \begin{center}
   \includegraphics[width=0.6\linewidth]{ut.jpg} \\
   Unreal Tournament 0.3.0-SVN
   \end{center}
}

\frame{
  \frametitle{Ende}
  \begin{center}
    Danke f�r Ihre Aufmerksamkeit\\
  \end{center}
}


\end{document}
