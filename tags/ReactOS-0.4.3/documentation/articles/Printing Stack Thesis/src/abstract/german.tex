Das Open-Source ReactOS Betriebssystem hat das Ziel, eine Alternative zum aktuell marktführenden PC-Betriebssystem Microsoft Windows zu schaffen, indem es volle Kompatibilität mit dafür geschriebenen Anwendungen und Treibern bietet.
Aus diesem Grund müssen existierende Anwendungen (z.B. Textverarbeitungsprogramme) in der Lage sein, die etablierten API-Funktionen zum Drucken auch unter ReactOS zu benutzen, ohne diese Anwendungen neu zu kompilieren oder deren Code zu verändern.

Diese Arbeit liefert eine ausführliche Recherche zum Druckerstack des Microsoft Windows Betriebssystems als auch einen Vergleich zu anderen verbreiteten Druckerstacks.
Es folgt der Entwurf und eine erste Implementierung von kompatiblen Komponenten für das ReactOS Betriebssystem, um das Drucken von vorbereiteten Daten in einer Druckerkontrollsprache unter Benutzung der etablierten Betriebssystem API-Funktionen zu ermöglichen.
Dies umfasst sowohl das Drucken auf echten Drucker, die physisch an einen Computer angeschlossen sind, als auch die Verwendung virtueller Drucker zur Ausgabe des Druckergebnisses in einer Datei.
Die Komponenten sind flexibel entworfen, sodass eine spätere Erweiterung um Treiberunterstützung für Datentyp-Konvertierungen, Benutzeroberflächenkomponenten und zusätzliche Druckertreiber möglich ist.
